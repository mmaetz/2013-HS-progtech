\usepackage{pifont}
\usepackage{ifthen}
\usepackage{blindtext}
\usepackage[splitindex]{imakeidx}%[2012/05/09]
\usepackage{listings}
\usepackage{tikz,tikz-3dplot,pgflibraryshapes}
%\usepackage[headings,cm]{fullpage}
\usetikzlibrary{positioning,calc,matrix,chains,scopes,fit,decorations,decorations.pathmorphing,decorations.pathreplacing,arrows,patterns,3d}
%%%\usepackage{pgfplots}
\usepackage[utf8]{inputenc}
\usepackage[ngerman,english]{babel}
\usepackage{newtxtext,newtxmath}

%\usepackage{epsfig,cite,amsmath,amssymb,amsbsy,mathrsfs}
\usepackage{epsfig,cite,amsmath,amssymb,amsbsy,mathrsfs}
\usepackage{graphicx}
\usepackage{color}
%\usepackage{latexsym}
%%% mmaetz packages
%%% Side captions
\usepackage{sidecap}
%%% Theorems like Theorem 3.1, proof environment
%\usepackage{amsthm}
%% NTheorem is a reimplementation of the AMS Theorem package. This will allow
%% us to typeset theorems like examples, proofs and similar.
%% NOTE: Must be loaded AFTER amsmath, or the \qed placement will break
%\usepackage[amsthm,thmmarks]{ntheorem}
%%% Using this packages numbers with units are always typed correctly.
\usepackage{siunitx}
%%% To cross out some stuff.
\usepackage{cancel}
%%% To have an enumerate environment with options to get like.
%%% i) bla
%%% ii) lala
%%% typing just \item
\usepackage{enumerate}
%%% To use \bm\mathrm instead of \vec one needs the appropriate greek letters
%\usepackage{upgreek}
%%% bold math
\usepackage{bm}
%%% Nice tabular
\usepackage{booktabs}
%%% For iddots (Inverse diagonal triple dot.)
%\usepackage{mathdots}
%%% To change the headers. 
\newcommand{\changefont}{%
	\fontsize{9}{11}\selectfont
}
%%% fancy headers
%\usepackage{fancyhdr}
%\pagestyle{fancy}
%\fancyhead[LE]{\changefont\slshape\nouppercase \rightmark} %section
%\fancyhead[RE]{\thepage}
%\fancyhead[RO]{\changefont\slshape\nouppercase \leftmark} % chapter
%\fancyhead[LO]{\thepage}
%\fancyfoot[C]{}

%%% To use mathclap.
\usepackage{mathtools}
%%% To have a set of bigger braces.
%%% I found out the guy who made that was at at the university of Lille, the city I was born! =)
\usepackage{yhmath}
%%% subfigures, subfloat used in week8.
\usepackage{subfig}

%%% mmaetz: This is for the change notes
%\usepackage[deletedmarkup=none]{changes}

%%% Use this option instead of the above one to make the red stuff disappear and the blue stuff become black. (And also remove the footnotes done with the change package.)
%%% Below are the options used for the change package. Note that the default is to cross out the deleted stuff but this doesn't work well in a math environment so the default settings have been changed.
%%% They have been commented out because all change markups have been removed.
%\usepackage[deletedmarkup=none]{changes}
%\setdeletedmarkup{\textcolor{red!75!black}{#1}}
%\setauthormarkupposition{left}
%\setremarkmarkup{\footnote{#1: \textcolor{Changes@Color#1}{#2}}}
%\setremarkmarkup{\marginpar{#1:#2}}
%%% mmaetz: To suppress the change notes put the final option like that:
%\usepackage[final]{changes}
%%% mmaetz: I'm using an authors id
%\definechangesauthor[name={Marc Maetz},color=blue!50!black]{MM}

%%% mmaetz: Can be useful for something but forgot what and I'm not using it here.
%\usepackage{etoolbox}
%%% mmaetz: Just discovered breqn which provides automatic line breaking. Very nice, more powerful but requires a bit of getting used to it.
%%% flexisym is needed by breqn
%\usepackage{flexisym}
%\usepackage{breqn}
%%% To make the references clickable
%%% The color settings are in the tikzstuff.tex file.
\usepackage{hyperref}
