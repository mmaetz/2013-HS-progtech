\chapter{Introduction}
\section{A first C++ program}
\begin{lstlisting}
/* A first program */
#include <iostream>
using namespace std;
int main()
{
cout << "Hello students\n";
// std::cout without the using declaration 
return 0;
}
\end{lstlisting}
\begin{itemize}
	\item /* and */ are the delimiters for
	comments
	\item includes declarations of I/O
	streams
	\item declares that we want to use the
	standard library (“std”)
	\item the main program is always
	called “main”
	\item “cout” is the standard output
	stream.
	\item “<<“ is the operator to write to a
	stream
	\item statements end with a ;
	\item // starts one-line comments
	\item A return value of 0 means that
	everything went OK
	1
\end{itemize}
\section{Namespace}
\begin{itemize}
	\item  All these versions are equivalent
	\item  Feel free to use any style in your
	program
	\item  Never use using statements
globally in libraries
\end{itemize}
\cht{namespace}
\begin{lstlisting}
	#include <iostream>
	using namespace std;
	int main()
	{
	cout << "Hello\n";
	}
\end{lstlisting}
\begin{lstlisting}
	#include <iostream>
	using std::cout;
	int main()
	{
	cout << "Hello\n";
	}
\end{lstlisting}
\begin{lstlisting}
	#include <iostream>
	int main()
	{
	cout << "Hello\n";
	}
\end{lstlisting}
\begin{lstlisting}
	 #include <iostream>
	 #include <cmath>
	 using namespace std;
	 int main()
	 {
	 cout << "The square root of 5 is"
	 << sqrt(5.) << "\n";
	 return 0;
	 }
\end{lstlisting}
\begin{itemize}
	\item \textless \index[cpp]{cmath}\textgreater is the header for
 mathematical functions
 \item Output can be connected by \textless\textless
 \item Expressions can be used in
 output statements
\end{itemize}
\section{Integral data types}
\begin{itemize}
	\item  Signed data types
		\begin{itemize}
			\item short, int, long, long long
			\item since C++11: int8_t, int16_t, int32_t, int64_t
		\end{itemize}
	\item  Unsigned data types
		\begin{itemize}
			\item unsigned short, unsigned int,
			unsigned long, unsigned long long
			\item since C++11: uint8_t, uint16_t, uint32_t, uint64_t
		\end{itemize}
	\item  Are stored as binary numbers of length
		\begin{itemize}
			\item short: usually 16 bit
			\item int: usually 32 bit
			\item long: usually 32 bit on 32-bit CPUs and 64 bit on 64-bit CPUs
			\item long long: usually 64 bits
		\end{itemize}
\end{itemize}
\section{Integer representations}
\begin{itemize}
	\item  An n-bit integer is stored in n/8 bytes
		\begin{itemize}
			\item Little-endian: least significant byte first
			\item Big-endian: most significant byte first
			\item Exercise: write a program to check the format of your CPU
		\end{itemize}
	\item  Unsigned
		\begin{itemize}
			\item x just stored as n bits, values from 0 \ldots 2 n -1
		\end{itemize}
	\item  Signed
		\begin{itemize}
			\item Stored as 2’s complement, values from -2 n-1 \ldots 2 n-1 -1
			\item Highest bit is sign S
			\item $x \geq 0$ : S=0, rest is $x$
			\item $x < 0$ : S=1, rest is $\sim (-x -1)$
			\item Advantage of this format: signed numbers can be added like unsigned
		\end{itemize}
\end{itemize}
