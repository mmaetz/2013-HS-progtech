%% Memoir layout setup
\makeindex[name=cpp,title=Index of C++]
\makeindex[name=theory,title=Index of theory]
\makeindex[name=git,title=Index of git]
\makeindex[name=makefile,title=Index of Makefile]

%% NOTE: You are strongly advised not to change any of them unless you
%% know what you are doing.  These settings strongly interact in the
%% final look of the document.

% Dependencies
\usepackage{array}
\counterwithout{chapter}{section}

% Define the default sans serif font as the lighter computer modern bright by
% D. Knuth.
\renewcommand{\sfdefault}{cmbr}

%%% Define a nice orange color and use it for hyperref
%%% Needs xcolor
\usepackage{xcolor}
\definecolor{rioday}{RGB}{255,166,0} 
\hypersetup{colorlinks=false,linkbordercolor=rioday}


% Turn extra space before chapter headings off.

% Chapter style redefinition
\makeatletter
\newcommand\thickhrulefill{\leavevmode \leaders%
\hrule height 6.25pt depth -3.25pt \hfill \kern \z@}
\setlength\midchapskip{10pt}
\makechapterstyle{VZ14}{
  \renewcommand\chapternamenum{}
  \renewcommand\printchaptername{}
  \renewcommand\chapnamefont{\Large\scshape}
  \renewcommand\printchapternum{%
    \chapnamefont\null\thickhrulefill\quad
    \@chapapp\space\thechapter\quad\thickhrulefill}
  \renewcommand\printchapternonum{%
    \par\thickhrulefill\par\vskip\midchapskip
    \hrule\vskip\midchapskip
  }
  \renewcommand\chaptitlefont{\Huge\scshape\centering}
  \renewcommand\afterchapternum{%
    \par\nobreak\vskip\midchapskip\hrule\vskip\midchapskip}
  \renewcommand\afterchaptertitle{%
    \par\vskip\midchapskip
\hrule\nobreak\vskip\afterchapskip}
}

% Set the way pages are layed out (headers and page numbering)
\pagestyle{ruled}
%\if@twoside
  %\pagestyle{Ruled}
%\else
  %\pagestyle{ruled}
%\fi

% Use the newly defined style
\chapterstyle{VZ14}

% Redefine sectional headings to contain rules
%\renewcommand{\section}{\@startsection{section}{1}{0mm}%
%{-2\baselineskip}{0.8\baselineskip}%
%{\hrule depth 0.2pt width\textwidth\hrule depth1.5pt%
%width0.25\textwidth\vspace*{1.2em}\Large\bfseries\sffamily}}

%\renewcommand{\subsection}{\@startsection{subsection}{2}{0mm}%
%{-2\baselineskip}{0.8\baselineskip}%
%{\hrule depth 0.2pt width\textwidth\hrule depth1pt width0.25\textwidth\vspace*{0.8em}\large\bfseries\sffamily}}

%\renewcommand{\subsubsection}{\@startsection{subsubsection}{3}{0mm}%
%{-2\baselineskip}{0.8\baselineskip}%
%{\large\bfseries\sffamily}}

\setparaheadstyle{\normalsize\bfseries\sffamily}
\setsubparaheadstyle{\normalsize\bfseries\sffamily}

% Set captions to a more separated style for clearness
\captionnamefont{\sffamily\bfseries\footnotesize}
\captiontitlefont{\sffamily\footnotesize}
\setlength{\intextsep}{16pt}
\setlength{\belowcaptionskip}{1pt}

%%% Make a bit of additional space for footnotes
\addtolength{\skip\footins}{4pt}
\renewcommand{\footnoterule}{%
   \kern -7pt                   % call this kerna
   \hrule height 0.4pt width 0.4\columnwidth
   \kern 6.6pt                  % call this kernb
}

% Set section and TOC numbering depth to subsection
\setsecnumdepth{subsection}
\settocdepth{subsection}

% Turn off american style paragraph indentation and add some space to be
% printed when a new paragraph starts.

\setlength{\parindent}{0pt}
\addtolength{\parskip}{2pt}

\newcommand{\professor}[1]{\def\@professor{#1}}
\renewcommand{\maketitlehookb}%
{\vspace{2em}\centering\Large\@professor\vspace{0.3\textheight}}

%% This provides a frontend to set the lecture date into the header
%% The chapter names are usually shorter than the section names. So the date should be at this place.
%\newcommand{\lecturedate}[1]{\def\@lecdate{#1}}
%\makeevenhead{ruled}{\normalfont\leftmark,}{}{\@lecdate}
%%% Make the header the same width as the text
%\makerunningwidth{ruled}{\textwidth}
%\makeheadrule{ruled}{\textwidth}{\normalrulethickness}
\renewcommand{\footruleskip}{-5pt}
\makeatother

% This defines how theorems should look. Best leave as is.
%\theoremstyle{plain}
%\theoremseparator{:\quad}
%\theoremprework{}
%\theoremindent2em
%\theoremheaderfont{\sffamily\bfseries}
%\theorembodyfont{\normalfont}
%\theoremsymbol{}
%% Minimal margin to print two pages on an A4 paper or viewing it on tablets.
\settypeblocksize{17.7cm}{11.8cm}{*}
\setlrmargins{2cm}{*}{*}
\setulmargins{1.6cm}{*}{*}
\setheadfoot{7pt}{20pt}
\setlength{\beforechapskip}{-1.2cm}
\checkandfixthelayout

 \definecolor{mygreen}{rgb}{0,0.6,0}
 \definecolor{mygray}{rgb}{0.5,0.5,0.5}
 \definecolor{mymauve}{rgb}{0.58,0,0.82}

 \lstset{ %
	otherkeywords={::},
  backgroundcolor=\color{white},   % choose the background color; you must add \usepackage{color} or \usepackage{xcolor}
  basicstyle=\footnotesize,        % the size of the fonts that are used for the code
  breakatwhitespace=false,         % sets if automatic breaks should only happen at whitespace
  breaklines=true,                 % sets automatic line breaking
  captionpos=b,                    % sets the caption-position to bottom
  commentstyle=\color{mygreen},    % comment style
  deletekeywords={...},            % if you want to delete keywords from the given language
  escapeinside={\%*}{*)},          % if you want to add LaTeX within your code
  extendedchars=true,              % lets you use non-ASCII characters; for 8-bits encodings only, does not work with UTF-8
  frame=single,                    % adds a frame around the code
  keepspaces=true,                 % keeps spaces in text, useful for keeping indentation of code (possibly needs columns=flexible)
  keywordstyle=\color{blue},       % keyword style
  language=C++,                 % the language of the code
  morekeywords={*,...},            % if you want to add more keywords to the set
  numbers=left,                    % where to put the line-numbers; possible values are (none, left, right)
  numbersep=5pt,                   % how far the line-numbers are from the code
  numberstyle=\tiny\color{mygray}, % the style that is used for the line-numbers
  rulecolor=\color{black},         % if not set, the frame-color may be changed on line-breaks within not-black text (e.g. comments (green here))
  showspaces=false,                % show spaces everywhere adding particular underscores; it overrides 'showstringspaces'
  showstringspaces=false,          % underline spaces within strings only
  showtabs=false,                  % show tabs within strings adding particular underscores
  stepnumber=1,                    % the step between two line-numbers. If it's 1, each line will be numbered
  stringstyle=\color{mymauve},     % string literal style
  tabsize=2,                       % sets default tabsize to 2 spaces
  title=\lstname                   % show the filename of files included with \lstinputlisting; also try caption instead of title
}
