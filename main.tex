%%%\documentclass[paper,notoc,12pt]{JHEP3}
%%%mmaetz: put a4paper
%\documentclass[a5paper]{book}
\documentclass[10pt,a5paper,twoside]{memoir}
%%%mmaetz: reduce margins
%%%mmaetz: omitted temporarily to have the margins for the change remarks.
%\usepackage[a5paper,left=2cm, right=1cm,top=1.5cm, bottom=1.5cm]{geometry}
%mmaetz: use tikz package
\usepackage{currfile}
\usepackage{pifont}
\usepackage{ifthen}
\usepackage{blindtext}
\usepackage[splitindex]{imakeidx}%[2012/05/09]
\usepackage{listings}
\usepackage{tikz,tikz-3dplot,pgflibraryshapes}
%\usepackage[headings,cm]{fullpage}
\usetikzlibrary{positioning,calc,matrix,chains,scopes,fit,decorations,decorations.pathmorphing,decorations.pathreplacing,arrows,patterns,3d}
%%%\usepackage{pgfplots}
\usepackage[utf8]{inputenc}
\usepackage[ngerman,english]{babel}

%\usepackage{epsfig,cite,amsmath,amssymb,amsbsy,mathrsfs}
\usepackage{epsfig,cite,amsmath,amssymb,amsbsy,mathrsfs}
\usepackage{graphicx}
\usepackage{color}
\usepackage{latexsym}
%%% mmaetz packages
%%% Side captions
\usepackage{sidecap}
%%% Theorems like Theorem 3.1, proof environment
%\usepackage{amsthm}
%% NTheorem is a reimplementation of the AMS Theorem package. This will allow
%% us to typeset theorems like examples, proofs and similar.
%% NOTE: Must be loaded AFTER amsmath, or the \qed placement will break
\usepackage[amsthm,thmmarks]{ntheorem}
%%% Using this packages numbers with units are always typed correctly.
\usepackage{siunitx}
%%% To cross out some stuff.
\usepackage{cancel}
%%% To have an enumerate environment with options to get like.
%%% i) bla
%%% ii) lala
%%% typing just \item
\usepackage{enumerate}
%%% To use \bm\mathrm instead of \vec one needs the appropriate greek letters
\usepackage{upgreek}
%%% bold math
\usepackage{bm}
%%% Nice tabular
\usepackage{booktabs}
%%% For iddots (Inverse diagonal triple dot.)
\usepackage{mathdots}
%%% To change the headers. 
\newcommand{\changefont}{%
	\fontsize{9}{11}\selectfont
}
%%% fancy headers
%\usepackage{fancyhdr}
%\pagestyle{fancy}
%\fancyhead[LE]{\changefont\slshape\nouppercase \rightmark} %section
%\fancyhead[RE]{\thepage}
%\fancyhead[RO]{\changefont\slshape\nouppercase \leftmark} % chapter
%\fancyhead[LO]{\thepage}
%\fancyfoot[C]{}

%%% To use mathclap.
\usepackage{mathtools}
%%% To have a set of bigger braces.
%%% I found out the guy who made that was at at the university of Lille, the city I was born! =)
\usepackage{yhmath}
%%% subfigures, subfloat used in week8.
\usepackage{subfig}

%%% mmaetz: This is for the change notes
%\usepackage[deletedmarkup=none]{changes}

%%% Use this option instead of the above one to make the red stuff disappear and the blue stuff become black. (And also remove the footnotes done with the change package.)
%%% Below are the options used for the change package. Note that the default is to cross out the deleted stuff but this doesn't work well in a math environment so the default settings have been changed.
%%% They have been commented out because all change markups have been removed.
%\usepackage[deletedmarkup=none]{changes}
%\setdeletedmarkup{\textcolor{red!75!black}{#1}}
%\setauthormarkupposition{left}
%\setremarkmarkup{\footnote{#1: \textcolor{Changes@Color#1}{#2}}}
%\setremarkmarkup{\marginpar{#1:#2}}
%%% mmaetz: To suppress the change notes put the final option like that:
%\usepackage[final]{changes}
%%% mmaetz: I'm using an authors id
%\definechangesauthor[name={Marc Maetz},color=blue!50!black]{MM}

%%% mmaetz: Can be useful for something but forgot what and I'm not using it here.
%\usepackage{etoolbox}
%%% mmaetz: Just discovered breqn which provides automatic line breaking. Very nice, more powerful but requires a bit of getting used to it.
%%% flexisym is needed by breqn
\usepackage{flexisym}
\usepackage{breqn}
%%% To make the references clickable
%%% The color settings are in the tikzstuff.tex file.
\usepackage{hyperref}

\newtheorem{expl}{Example}[chapter]
\newtheorem{remk}{Remark}[chapter]
\newtheorem{corl}{Corollary}[chapter]
\newtheorem{thrm}{Theorem}[chapter]
\newtheorem{exer}{Exercise}[chapter]

%% Proof environment with a small square as a "qed" symbol
%\theoremstyle{nonumberplain}
%\newtheorem{proof}{Proof}
%\qedsymbol{\qed}
%\theoremsymbol{\qed}

%% Memoir layout setup
\makeindex[name=cpp,title=Index of C++]
\makeindex[name=theory,title=Index of theory]
\makeindex[name=git,title=Index of git]
\makeindex[name=makefile,title=Index of Makefile]

%% NOTE: You are strongly advised not to change any of them unless you
%% know what you are doing.  These settings strongly interact in the
%% final look of the document.

% Dependencies
\usepackage{array}
\counterwithout{chapter}{section}

% Define the default sans serif font as the lighter computer modern bright by
% D. Knuth.
\renewcommand{\sfdefault}{cmbr}

%%% Define a nice orange color and use it for hyperref
%%% Needs xcolor
\usepackage{xcolor}
\definecolor{rioday}{RGB}{255,166,0} 
\hypersetup{colorlinks=false,linkbordercolor=rioday}


% Turn extra space before chapter headings off.

% Chapter style redefinition
\makeatletter
\newcommand\thickhrulefill{\leavevmode \leaders%
\hrule height 6.25pt depth -3.25pt \hfill \kern \z@}
\setlength\midchapskip{10pt}
\makechapterstyle{VZ14}{
  \renewcommand\chapternamenum{}
  \renewcommand\printchaptername{}
  \renewcommand\chapnamefont{\Large\scshape}
  \renewcommand\printchapternum{%
    \chapnamefont\null\thickhrulefill\quad
    \@chapapp\space\thechapter\quad\thickhrulefill}
  \renewcommand\printchapternonum{%
    \par\thickhrulefill\par\vskip\midchapskip
    \hrule\vskip\midchapskip
  }
  \renewcommand\chaptitlefont{\Huge\scshape\centering}
  \renewcommand\afterchapternum{%
    \par\nobreak\vskip\midchapskip\hrule\vskip\midchapskip}
  \renewcommand\afterchaptertitle{%
    \par\vskip\midchapskip
\hrule\nobreak\vskip\afterchapskip}
}

% Set the way pages are layed out (headers and page numbering)
\pagestyle{ruled}
%\if@twoside
  %\pagestyle{Ruled}
%\else
  %\pagestyle{ruled}
%\fi

% Use the newly defined style
\chapterstyle{VZ14}

% Redefine sectional headings to contain rules
%\renewcommand{\section}{\@startsection{section}{1}{0mm}%
%{-2\baselineskip}{0.8\baselineskip}%
%{\hrule depth 0.2pt width\textwidth\hrule depth1.5pt%
%width0.25\textwidth\vspace*{1.2em}\Large\bfseries\sffamily}}

%\renewcommand{\subsection}{\@startsection{subsection}{2}{0mm}%
%{-2\baselineskip}{0.8\baselineskip}%
%{\hrule depth 0.2pt width\textwidth\hrule depth1pt width0.25\textwidth\vspace*{0.8em}\large\bfseries\sffamily}}

%\renewcommand{\subsubsection}{\@startsection{subsubsection}{3}{0mm}%
%{-2\baselineskip}{0.8\baselineskip}%
%{\large\bfseries\sffamily}}

\setparaheadstyle{\normalsize\bfseries\sffamily}
\setsubparaheadstyle{\normalsize\bfseries\sffamily}

% Set captions to a more separated style for clearness
\captionnamefont{\sffamily\bfseries\footnotesize}
\captiontitlefont{\sffamily\footnotesize}
\setlength{\intextsep}{16pt}
\setlength{\belowcaptionskip}{1pt}

%%% Make a bit of additional space for footnotes
\addtolength{\skip\footins}{4pt}
\renewcommand{\footnoterule}{%
   \kern -7pt                   % call this kerna
   \hrule height 0.4pt width 0.4\columnwidth
   \kern 6.6pt                  % call this kernb
}

% Set section and TOC numbering depth to subsection
\setsecnumdepth{subsection}
\settocdepth{subsection}

% Turn off american style paragraph indentation and add some space to be
% printed when a new paragraph starts.

\setlength{\parindent}{0pt}
\addtolength{\parskip}{2pt}

\newcommand{\professor}[1]{\def\@professor{#1}}
\renewcommand{\maketitlehookb}%
{\vspace{2em}\centering\Large\@professor\vspace{0.3\textheight}}

%% This provides a frontend to set the lecture date into the header
%% The chapter names are usually shorter than the section names. So the date should be at this place.
%\newcommand{\lecturedate}[1]{\def\@lecdate{#1}}
%\makeevenhead{ruled}{\normalfont\leftmark,}{}{\@lecdate}
%%% Make the header the same width as the text
%\makerunningwidth{ruled}{\textwidth}
%\makeheadrule{ruled}{\textwidth}{\normalrulethickness}
\renewcommand{\footruleskip}{-5pt}
\makeatother

% This defines how theorems should look. Best leave as is.
\theoremstyle{plain}
\theoremseparator{:\quad}
\theoremprework{}
\theoremindent2em
\theoremheaderfont{\sffamily\bfseries}
\theorembodyfont{\normalfont}
%\theoremsymbol{}
%% Minimal margin to print two pages on an A4 paper or viewing it on tablets.
\settypeblocksize{17.7cm}{11.8cm}{*}
\setlrmargins{2cm}{*}{*}
\setulmargins{1.6cm}{*}{*}
\setheadfoot{7pt}{20pt}
\setlength{\beforechapskip}{-1.2cm}
\checkandfixthelayout

 \definecolor{mygreen}{rgb}{0,0.6,0}
 \definecolor{mygray}{rgb}{0.5,0.5,0.5}
 \definecolor{mymauve}{rgb}{0.58,0,0.82}

 \lstset{ %
	otherkeywords={::},
  backgroundcolor=\color{white},   % choose the background color; you must add \usepackage{color} or \usepackage{xcolor}
  basicstyle=\footnotesize,        % the size of the fonts that are used for the code
  breakatwhitespace=false,         % sets if automatic breaks should only happen at whitespace
  breaklines=true,                 % sets automatic line breaking
  captionpos=b,                    % sets the caption-position to bottom
  commentstyle=\color{mygreen},    % comment style
  deletekeywords={...},            % if you want to delete keywords from the given language
  escapeinside={\%*}{*)},          % if you want to add LaTeX within your code
  extendedchars=true,              % lets you use non-ASCII characters; for 8-bits encodings only, does not work with UTF-8
  frame=single,                    % adds a frame around the code
  keepspaces=true,                 % keeps spaces in text, useful for keeping indentation of code (possibly needs columns=flexible)
  keywordstyle=\color{blue},       % keyword style
  language=C++,                 % the language of the code
  morekeywords={*,...},            % if you want to add more keywords to the set
  numbers=left,                    % where to put the line-numbers; possible values are (none, left, right)
  numbersep=5pt,                   % how far the line-numbers are from the code
  numberstyle=\tiny\color{mygray}, % the style that is used for the line-numbers
  rulecolor=\color{black},         % if not set, the frame-color may be changed on line-breaks within not-black text (e.g. comments (green here))
  showspaces=false,                % show spaces everywhere adding particular underscores; it overrides 'showstringspaces'
  showstringspaces=false,          % underline spaces within strings only
  showtabs=false,                  % show tabs within strings adding particular underscores
  stepnumber=2,                    % the step between two line-numbers. If it's 1, each line will be numbered
  stringstyle=\color{mymauve},     % string literal style
  tabsize=2,                       % sets default tabsize to 2 spaces
  title=\lstname                   % show the filename of files included with \lstinputlisting; also try caption instead of title
}

%% Custom commands =============================================================
\newcommand{\ifcontent}[1]{\ifthenelse{\equal{#1}{}}{}{\addcontentsline{toc}{section}{#1}}}

\newcommand{\cht}[1]{\index[cpp]{#1}}
\newcommand{\tht}[1]{\index[theory]{#1}}
\newcommand{\ght}[1]{\index[git]{#1}}
\newcommand{\mht}[1]{\index[makefile]{#1}}
\newcommand{\psd}[7]{\begin{center}%
	\ifcontent{#5}
	\includegraphics[page=#2,width=#4\textwidth]{../mod-slides/#1.pdf}\\%
	\ifcontent{#6}
	\includegraphics[page=#3,width=#4\textwidth]{../mod-slides/#1.pdf}%
\end{center}#7}
\newcommand{\pss}[5]{\begin{center}%
	\ifcontent{#4}
	\includegraphics[page=#2,width=#3\textwidth]{../mod-slides/#1.pdf}\\%
\end{center}#5}

%% Depends on this
\usepackage{ifthen}
\usepackage{rotating}
%\usepackage[all]{xy}

%% Differential d. This sets it as proper operator in roman type. With correct
%% spacing. ISO standards for mathematical typesetting says it should be printed
%% like this.
%\newcommand{\diff}[1]{\operatorname{d}\ifthenelse{\equal{#1}{}}{\,}{\!#1}}
\newcommand{\dd}[1]{\operatorname{d}\ifthenelse{\equal{#1}{}}{\,}{\!#1}}
\newcommand{\rd}{\mathrm d}

%% Symbols for euler number and imaginary unit
\providecommand*{\eu}%
{\ensuremath{\mathrm{e}}}
% The imaginary unit
\providecommand*{\iu}%
{\ensuremath{\mathrm{i}}} % i can be replaced with j on preference.

%% Uncomment below what style you prefer for printing differential operators
%\DeclareMathOperator{\grad}{grad}
%\DeclareMathOperator{\rot}{rot}
%\DeclareMathOperator{\Div}{div}
\DeclareMathOperator{\grad}{\nabla}
\DeclareMathOperator{\rot}{\nabla\times}
\DeclareMathOperator{\Div}{\nabla\cdot}

%% Additional mathematical operators
%% Just use the physics package
\DeclareMathOperator{\sgn}{tr}
\DeclareMathOperator{\tr}{tr}
\DeclareMathOperator{\id}{Id}
\DeclareMathOperator{\arccot}{arccot}
\DeclareMathOperator{\arsinh}{arsinh}
\DeclareMathOperator{\arcosh}{arcosh}
\DeclareMathOperator{\artanh}{artanh}
%% German variants
%\DeclareMathOperator{\Kern}{Kern}
%\DeclareMathOperator{\Bild}{Bild}
%\DeclareMathOperator{\Grad}{Grad}
%% English variants
%% \ker is provided by LaTeX
\DeclareMathOperator{\im}{im}
%% \grad is provided by LaTeX

%% Special characters for number sets, e.g. real or complex numbers.
\newcommand{\C}{\mathbb{C}}
\newcommand{\K}{\mathbb{K}}
\newcommand{\N}{\mathbb{N}}
\newcommand{\Q}{\mathbb{Q}}
\newcommand{\R}{\mathbb{R}}
\newcommand{\Z}{\mathbb{Z}}
\newcommand{\X}{\mathbb{X}}

%% Fixed size delimiter examples
\newcommand{\floor}[1]{\lfloor #1 \rfloor}
\newcommand{\ceil}[1]{\lceil #1 \rceil}
\newcommand{\seq}[1]{\langle #1 \rangle}
\newcommand{\set}[1]{\{ #1 \}}
\newcommand{\abs}[1]{\left\lvert #1 \right\rvert}
\newcommand{\norm}[1]{\left\lVert #1 \right\rVert}
\newcommand{\indic}[1]{\bigl[#1\bigr]}
\newcommand{\comm}[2]{\left[ {#1}, {#2} \right] }
\newcommand{\acomm}[2]{\left\{ {#1}, {#2} \right\} }

%% Scaling delimiter examples
\newcommand{\Floor}[1]{\left\lfloor #1 \right\rfloor}
\newcommand{\Ceil}[1]{\left\lceil #1 \right\rceil}
\newcommand{\Seq}[1]{\left\langle #1 \right\rangle}
\newcommand{\Set}[1]{\left\{ #1 \right\}}
\newcommand{\Abs}[1]{\left\lvert #1 \right\rvert}
\newcommand{\Norm}[1]{\left\lVert #1 \right\rVert}

%% Absolute and partial derrivate fractions
%% Calculus
\newcommand{\dv}[2]{\frac{\mathrm d #1}{\mathrm d #2}}
\newcommand{\dvt}[3]{\frac{\mathrm d ^{2} #1}{\mathrm d #2 ^{2}}}
\newcommand{\dvn}[3]{\frac{\mathrm d ^{#1} #2}{\mathrm d #3 ^{#1}}}
\newcommand{\pdv}[2]{\frac{\partial #1}{\partial #2}}
\newcommand{\pddv}[2]{\frac{\partial^{2} #1}{\partial #2 ^{2}}}
\newcommand{\pddvm}[3]{\frac{\partial^{2} #1}{\partial #2 \partial #3}}
\newcommand{\pdvn}[3]{\frac{\partial^{#1} #2}{\partial #3 ^{#1}}}
\newcommand{\fdv}[2]{\frac{\delta #1}{\delta #2}}
\newcommand{\vtr}[1]{\boldsymbol{\mathrm{#1}}}


%% Set an index and print it to the current position at the same time
\newcommand{\Index}[1]{\emph{#1}\index{#1}}

%% Displaystyle math for inline math mode
\newcommand{\ds}{\displaystyle}

%% Easy to use alias for the default matrices with round braces
\newcommand{\Mx}[1]{\ensuremath{\begin{pmatrix}#1\end{pmatrix}}}

%% Include a lecture from the lectures/ folder by date.
%% Added ddmmyydate option because the default format is too large.


%% A macro to typeset a commutitive diagram in the style of
%% \[\Abb[functionname]{from}{to}{fromelement}{toelement}\]
\newcommand{\Sidein}{\begin{rotate}{90}\small$\in$\end{rotate}}
\newcommand{\sidew}[1]{\rotatebox{90}{\small$#1$}}

\newcommand{\Abb}[5][]{\ensuremath{
    \begin{array}{lc}
      \ifthenelse{\equal{#1}{}}{}{#1:}\;\; &
      \begin{xy}
        \xymatrixrowsep{1em}\xymatrixcolsep{2em}%
        \xymatrix{ #2 \ar[r] \ar@{}[d]^<<<<{\hspace{0.001em} \Sidein}
          & #3  \ar@{}[d]^<<<<{\hspace{0.001em} \Sidein} \\
          #4 \ar@{|->}[r] & #5} \end{xy}
    \end{array}
  }%
}


%% Use the alternative epsilon per default and define the old one as \oldepsilon
\let\oldepsilon\epsilon

\renewcommand{\epsilon}{\ensuremath\varepsilon}

%% Also set the alternate phi as default.
%\renewcommand{\phi}{\ensuremath{\varphi}}

%% Uncomment to type bra, kets etc.
\newcommand{\bra}[1]{\left\langle #1\right|}
\newcommand{\ket}[1]{\left| #1\right\rangle}
\newcommand{\braket}[2]{\left\langle #1\right|\left.\! #2\right\rangle}
\newcommand{\ketbra}[2]{\ensuremath{ {\ket{#1} \!\bra{#2}}}}
\newcommand{\proj}[1]{\ensuremath{ {\ket{#1} \!\bra{#1}}}}
\newcommand{\sumproj}[1]{\ensuremath{ {\sum_{#1}\proj{#1}}}}
%% matrix element
\newcommand{\mel}[3]{\ensuremath{\left\langle {#1}\vphantom{#3} \right\rvert{#2} \left\lvert{#3}\vphantom{#1}\right\rangle}}
\newcommand{\ev}[3]{\ensuremath{\left\langle {#1}\right\rvert{#2} \left\lvert{#1}\right\rangle}}

\newcommand{\vevj}[2]{\left\langle {#1} \right\langle_{#2} }
\newcommand{\vev}[1]{\left\langle {#1} \right\rangle }

\newcommand{\cmark}{{\color{black!30!green}\ding{51}}}
\newcommand{\xmark}{{\color{black!30!red}\ding{55}}}

%%% New definition of square root:
%%% it renames \sqrt as \oldsqrt
%\let\oldsqrt\sqrt
%%% it defines the new \sqrt in terms of the old one
%\def\sqrt{\mathpalette\DHLhksqrt}
%\def\DHLhksqrt#1#2{%
%	\setbox0=\hbox{$#1\oldsqrt{#2\,}$}\dimen0=\ht0
%	\advance\dimen0-0.2\ht0
%	\setbox2=\hbox{\vrule height\ht0 depth -\dimen0}%
%	{\box0\lower0.4pt\box2}}

%%% Theorems. (Copied from www.mitschriften.ethz.ch template)


\title{\Huge Programming Techniques for Scientific Simulations
}


\preauthor{}
\author{\LARGE Marc Maetz}
\postauthor{\Large\\\vspace{2em}
	Prof. Dr. Matthias Troyer
  %ETH Zurich,\\
  %8093 Zurich, Switzerland\\
	\vspace{1em}
  %E-mail: {\em babis@phys.ethz.ch}
}
\professor{}
%\professor{Prof. Dr. Markus Gross, Prof. Dr. Marc Pollefeys}

\date{\vspace{1em}\today}

%\author{Babis Anastasiou\\
%  Institute for Theoretical Physics, \\ 
%  ETH Zurich,\\
%  8093 Zurich, Switzerland\\
%  E-mail: {\em babis@phys.ethz.ch}}

\begin{document} 

\begin{titlingpage}
\maketitle
	%\titleS
%\vfill
%Revision \SVNRev{} --- \SVNDate{}
\end{titlingpage}
\nopagebreak[0]


%%% Color settings and plot settings. If the pictures look bad look at this file!
%\definecolor{scolor}{RGB}{255,220,181} 
%\definecolor{scolor}{RGB}{255,166,0} 
%\colorlet{scolordark}{scolor!50!black}

\definecolor{rioday}{RGB}{255,166,0} 
\colorlet{riomorning}{rioday!65!black}
\colorlet{rionight}{rioday!50!black}
%\definecolor{scolor2}{RGB}{0,255,166} 
%\definecolor{scolor2}{RGB}{100,100,170} 
%\definecolor{scolor2}{RGB}{0,233,255} 
%\definecolor{scolor2}{RGB}{0,182,255} 
%\definecolor{scolor2}{RGB}{0,255,16} 
%\definecolor{scolor2}{RGB}{0,89,255} 
%\colorlet{scolor2dark}{scolor2!50!black}

\definecolor{helsinkiday}{RGB}{0,89,255} 
%%% Hi from 2013. I think it's fascinating that this code could possibly be used until like 2043 or even further.
\colorlet{helsinkimorning}{helsinkiday!65!black}
\colorlet{helsinkinight}{helsinkiday!50!black}
\hypersetup{colorlinks=false,linkbordercolor=rioday}
%%%This is the global setting of the number of samples used to plot the functions in the graphs. 
\tikzset{
	smasa/.style={
		%%%The number of samples with low number.
		%%% 1000 should be enough for a final edition
		%samples=10
		samples=1000
	},
	bigsa/.style={
		%%%The number of samples with high number.
		%%% 10000 should be enough for a final edition.
		%samples=20
		samples=10000
	},
	%%%The standard filling
	sfill/.style={
		fill=rioday
	},
	zyplane/.style={canvas is zy plane at x=#1,very thin},
	zxplane/.style={canvas is zx plane at y=#1,very thin},
	yxplane/.style={canvas is yx plane at z=#1,very thin}
}




%\begin{abstract}
%The subject of the course an  introduction to  quantum 
%field theory. The following topics are discussed: 
%\begin{itemize}
%\item Theory of classical fields.  
%\item Canonical quantization of free fields.
%\item The  Dirac equation and  quantization of the Dirac field
%\item Field Propagation, interacting fields and perturbation theory. 
%\item Cross-sections  and decay rates.
%\item Introduction to QED and the problem of infinities.
%\item One-loop renormalization of QED. 
%\end{itemize}
%\end{abstract}

%\tableofcontents 
%\listofchanges
%\bibliographystyle{JHEP}
%\begin{thebibliography}{10}
%\bibitem{srednicki}
%Modern Quantum Mechanics, Sakurai 
%\bibitem{sterman}
%The Feynman Lectures in Physics, Feynman 
%\bibitem{weinberg}
%The Quantum Theory of Fields, Weinberg 
%\end{thebibliography} 
%\newpage

%%% Have decided to reverse the order of the material wrt the first time I taught this. 
%%% Files renamed according to the new order.

%%%%%%%%%%%%%%%%%%%%%%%%%%%%%%%%%%%%%%%%%%%%%%%%%
%%%%% bibliography
%%%%%%%%%%%%%%%%%%%%%%%%%%%%%%%%%%%%%%%%%%%%%%%%%%

% \pss{pdf file}{page}{zoom-factor}{title -> section into toc, not visible in text}{comments & \*ht's}
% \psd{pdf file}{page from}{page to}{zoom factor}{title of first page -> ..}{title -> ..}{comments & \*ht's}
% \*ht{text} not visible, only in index, see macrosetup.tex

	\chapter{Preface}
	\pss{week1-crop}{1}{0.96}{}{}
	\psd{week1-crop}{2}{3}{0.96}{}{}{}
	\psd{week1-crop}{4}{5}{0.96}{}{}{}
	\psd{week1-crop}{6}{7}{0.96}{Why C++}{Generic programming}{\cht{efficiency}}
	\psd{week1-crop}{8}{9}{0.96}{Programming language comparison}{}{\tht{Programming language comparison}}
	\psd{week1-crop}{10}{11}{0.96}{}{}{}
	\psd{week1-crop}{12}{13}{0.96}{Quiz}{}{\tht{quiz}}
	\psd{week1-crop}{14}{15}{0.96}{}{}{}
	\psd{week1-crop}{16}{17}{0.96}{Loop examples}{}{\cht{loop examples}}
	\psd{week1-crop}{18}{19}{0.96}{}{}{}
	\psd{week1-crop}{20}{21}{0.96}{}{}{}
	\pss{week1-crop}{22}{0.96}{}{}
\chapter{Version control}
\pss{week1-crop}{23}{0.96}{}{}
\psd{week1-crop}{24}{25}{0.96}{}{}{}
\psd{week1-crop}{26}{27}{0.96}{Version control systems}{Setting up git locally}{\tht{svn}\tht{cvs}\tht{git}\tht{mercurial (Hg)}}
\psd{week1-crop}{28}{29}{0.96}{Starting to work with git}{Add a file and commit it}{}
\psd{week1-crop}{30}{31}{0.96}{Local file, staging area and repository}{}{\ght{local file}\ght{staging area}\ght{repository}}
\psd{week1-crop}{32}{33}{0.96}{}{}{}
\psd{week1-crop}{34}{35}{0.96}{}{}{}
\psd{week1-crop}{36}{37}{0.96}{Collaborating needs a common repository}{Setting up the gitlab repository}{}
\psd{week1-crop}{38}{39}{0.96}{Pushing changes to the repository on the server}{Connecting the repositories}{\ght{push}}
\psd{week1-crop}{40}{41}{0.96}{Pulling changes and resolving conflicts}{Removing and unstaging files}{\ght{checkout}\ght{rm}}
\psd{week1-crop}{42}{43}{0.96}{Undoing changes}{Seeing differences to previous versions}{\ght{reset}\ght{diff}}
\psd{week1-crop}{44}{45}{0.96}{Branching}{Merging}{\ght{branch}\ght{merge}\ght{checkout}}
\psd{week1-crop}{46}{47}{0.96}{}{Advice}{}

\chapter{Introduction}
\pss{week2-crop}{1}{0.96}{}{}
\psd{week2-crop}{2}{3}{0.96}{A first C++ program}{Std namespace}{\cht{comments}\cht{iostream}\cht{std}\cht{std::cout}\cht{\textless\textless}\cht{//}\cht{namespace}\cht{using}}
\psd{week2-crop}{4}{5}{0.96}{A first calculation}{Integral data types}{\cht{short}\cht{int}\cht{long}\cht{long long}\cht{int8\_t}\cht{int16\_t}\cht{int32\_t}\cht{int64\_t}\cht{C++11}\cht{signed data types}\cht{Unsigned data types}\cht{unsigned short}\cht{unsigned int}\cht{unsigned long}\cht{unsigned long long}\cht{uint8\_t}\cht{uint16\_t}\cht{uint32\_t}\cht{uint64\_t}}
\psd{week2-crop}{6}{7}{0.96}{Integer representations}{Integer constants}{\tht{Little-endian}\tht{Big-endian}}
\psd{week2-crop}{8}{9}{0.96}{Characters}{Strings}{\cht{char}\cht{unsigned char}\cht{signed char}\cht{wchar\_t}\cht{string}}
\psd{week2-crop}{10}{11}{0.96}{Boolean (logical) type}{Floating point numbers }{\cht{bool}\cht{true}\cht{false}\cht{float}\cht{double}\cht{long double}\cht{single precision}\cht{double precision}\cht{extended precision}}
\psd{week2-crop}{12}{13}{0.96}{IEEE floating point representation}{Converting to/from IEEE representation}{\tht{IEEE representation}}
\psd{week2-crop}{14}{15}{0.96}{Floating point arithmetic}{Implementation-specific properties of numeric types}{\cht{truncation}\cht{machine precision}\cht{roundoff}\cht{limits}\cht{numeric\_limits}}
\psd{week2-crop}{16}{17}{0.96}{A more useful program}{Variable declarations}{\cht{variable declarations}\cht{declaration}\cht{initializer}}
\psd{week2-crop}{18}{19}{0.96}{Advanced types}{Expressions and operators}{\cht{enumerator}\cht{array}\cht{typedef}\cht{expressions}\cht{operators}}
\psd{week2-crop}{20}{21}{0.96}{Bitwise operations}{Compound assignments}{\cht{bitwise operations}\cht{compound assignments}}
\psd{week2-crop}{22}{23}{0.96}{Special operators}{Operator precedences}{\cht{special operators}\cht{operator precendences}}
\psd{week2-crop}{24}{25}{0.96}{Statements}{The if statement}{\cht{statements}\cht{if}\cht{else}}
\psd{week2-crop}{26}{27}{0.96}{The switch statement}{The for loop statement}{\cht{switch}\cht{for}}
\psd{week2-crop}{28}{29}{0.96}{The while statement}{the do-while statement}{\cht{while}\cht{do-while}}
\psd{week2-crop}{30}{31}{0.96}{The break and continue statements}{A loop example: what is wrong?}{\cht{break}\cht{continue}}
\psd{week2-crop}{32}{33}{0.96}{The goto statement}{Static memory allocation}{\cht{static memory allocation}\cht{memory allocation}}
\psd{week2-crop}{34}{35}{0.96}{Pointers}{Dynamic allocation}{\cht{pointers}\cht{dynamic allocation}\cht{memory allocation}}
\psd{week2-crop}{36}{37}{0.96}{Pointer arithmetic}{Arrays and pointers}{\cht{pointer arithmetic}\cht{arrays}\cht{pointers}}
\psd{week2-crop}{38}{39}{0.96}{A look at memory:array example}{A look at memory: pointer example}{}
\psd{week2-crop}{40}{41}{0.96}{References}{A more flexible program:function calls}{\cht{reference}\cht{function}}
\psd{week2-crop}{42}{43}{0.96}{Function call syntax}{pass by value}{\cht{function call}\cht{pass by value}}
\psd{week2-crop}{44}{45}{0.96}{pass by reference}{pass by const reference}{\cht{pass by reference}\cht{pass by const reference}}
\psd{week2-crop}{46}{47}{0.96}{Pass by pointer}{A swap example}{\cht{pass by pointer}\cht{swap}}
\psd{week2-crop}{48}{49}{0.96}{Type casts: static\_cast, reinterpret\_cast}{Type casts: const\_cast}{\cht{cast}\cht{static\_cast}\cht{reinterpret\_cast}}
\psd{week2-crop}{50}{51}{0.96}{Namespaces}{Default function arguments}{\cht{namespaces}}

\chapter{Preprocessor}
\pss{week3a-crop}{1}{0.96}{}
\psd{week3a-crop}{2}{3}{0.96}{Steps when compiling a program}{The C++ preprocessor}{\cht{preprocessor}}
\psd{week3a-crop}{4}{5}{0.96}{\#define}{\#undef}{\cht{\#define}\cht{\#undef}}
\psd{week3a-crop}{6}{7}{0.96}{Looking at preprocessor output}{\!ifdef \ldots \#endif}{\cht{preprocessor output}}
\psd{week3a-crop}{8}{9}{0.96}{\#if\ldots\#elif \ldots \#endif}{\#error}{\cht{conditional compilation}\cht{compilation}\cht{\#ifdef}\cht{\#endif}\cht{\#if}\cht{\#elif}\cht{\#endif}}
\psd{week3a-crop}{10}{11}{0.96}{\#include ``file.h''\quad \#include \textless iostream\textgreater}{Segmenting programs}{\cht{\#iostream}\cht{\#include}}
\psd{week3a-crop}{12}{13}{0.96}{Compiling and linking}{Include guards}{\cht{compiling}\cht{linking}}
\psd{week3a-crop}{14}{15}{0.96}{Assert in header \textless cassert\textgreater}{Making a library on Linux/Unix/MacOS X}{\cht{cassert}\cht{library}\cht{ranlib}\cht{ar}}
\psd{week3a-crop}{16}{17}{0.96}{How libraries work}{Documenting your library}{\cht{library}}
\psd{week3a-crop}{18}{19}{0.96}{Example documentation}{The cost of a function call}{\cht{function cost}}
\pss{week3a-crop}{20}{0.96}{Inlining}{\cht{inlining}}
\chapter{Automated Builds}
\pss{week3b-crop}{1}{0.96}{}{}
\psd{week3b-crop}{2}{3}{0.96}{A typical working day}{}{\mht{Start of Makefile lecture}}
\psd{week3b-crop}{4}{5}{0.96}{}{}{}
\psd{week3b-crop}{6}{7}{0.96}{}{Manage tasks and dependencies}{\mht{tasks}\mht{dependencies}}
\psd{week3b-crop}{8}{9}{0.96}{}{What is newer than what?}{}
\psd{week3b-crop}{10}{11}{0.96}{Put this in a \emph{Makefile} called hello.mk}{}{\mht{target}\mht{prerequisite}\mht{comment rule}\mht{action}}
\psd{week3b-crop}{12}{13}{0.96}{}{Run Make from the shell}{\mht{indentation}\mht{run}}
\psd{week3b-crop}{14}{15}{0.96}{}{}{}
\psd{week3b-crop}{16}{17}{0.96}{}{Usually have multiple targets per file}{\mht{multiple targets}}
\psd{week3b-crop}{18}{19}{0.96}{}{}{}
\psd{week3b-crop}{20}{21}{0.96}{introduce a phony target}{}{\mht{phony target}}
\psd{week3b-crop}{22}{23}{0.96}{Both targets and prerequeisites}{Generalize to any number of files}{\mht{directed graph}\mht{many files}}
\psd{week3b-crop}{24}{25}{0.96}{Automatic variable}{}{\mht{automatic variable}}
\psd{week3b-crop}{26}{27}{0.96}{<++>}{<++>}{<++>}
\psd{week3b-crop}{28}{29}{0.96}{<++>}{<++>}{<++>}
\psd{week3b-crop}{30}{31}{0.96}{<++>}{<++>}{<++>}
\psd{week3b-crop}{32}{33}{0.96}{<++>}{<++>}{<++>}
\psd{week3b-crop}{34}{35}{0.96}{<++>}{<++>}{<++>}
\psd{week3b-crop}{36}{37}{0.96}{<++>}{<++>}{<++>}
\psd{week3b-crop}{38}{39}{0.96}{<++>}{<++>}{<++>}
\psd{week3b-crop}{40}{41}{0.96}{<++>}{<++>}{<++>}
\psd{week3b-crop}{42}{43}{0.96}{<++>}{<++>}{<++>}
\psd{week3b-crop}{44}{45}{0.96}{<++>}{<++>}{<++>}
\psd{week3b-crop}{46}{47}{0.96}{<++>}{<++>}{<++>}
\psd{week3b-crop}{48}{49}{0.96}{<++>}{<++>}{<++>}
\psd{week3b-crop}{50}{51}{0.96}{<++>}{<++>}{<++>}
\psd{week3b-crop}{52}{53}{0.96}{<++>}{<++>}{<++>}
\psd{week3b-crop}{54}{55}{0.96}{<++>}{<++>}{<++>}
\psd{week3b-crop}{56}{57}{0.96}{<++>}{<++>}{<++>}
\psd{week3b-crop}{58}{58}{0.96}{<++>}{<++>}{<++>}
\psd{week3b-crop}{60}{61}{0.96}{<++>}{<++>}{<++>}
\pss{week3b-crop}{62}{0.96}{<++>}{<++>}

\chapter{Introduction to hardware of the PC}
\pss{week4-crop}{1}{0.96}{}
\psd{week4-crop}{2}{3}{0.96}{}
\psd{week4-crop}{4}{5}{0.96}{}
\psd{week4-crop}{6}{7}{0.96}{}
\psd{week4-crop}{8}{9}{0.96}{}
\psd{week4-crop}{10}{11}{0.96}{}
\psd{week4-crop}{12}{13}{0.96}{}
\psd{week4-crop}{14}{15}{0.96}{}
\psd{week4-crop}{16}{17}{0.96}{}
\psd{week4-crop}{18}{19}{0.96}{}
\psd{week4-crop}{20}{21}{0.96}{}
\psd{week4-crop}{22}{23}{0.96}{}
\psd{week4-crop}{24}{25}{0.96}{}
\psd{week4-crop}{26}{27}{0.96}{}
\psd{week4-crop}{28}{29}{0.96}{}
\psd{week4-crop}{30}{31}{0.96}{}
\pss{week4-crop}{32}{0.96}{}

\chapter{Templates and generic programming}
\pss{week5a-crop}{1}{0.96}{<++>}{<++>}
\psd{week5a-crop}{2}{3}{0.96}{<++>}{<++>}{<++>}
\psd{week5a-crop}{4}{5}{0.96}{<++>}{<++>}{<++>}
\psd{week5a-crop}{6}{7}{0.96}{<++>}{<++>}{<++>}
\psd{week5a-crop}{8}{9}{0.96}{<++>}{<++>}{<++>}
\psd{week5a-crop}{10}{11}{0.96}{<++>}{<++>}{<++>}
\psd{week5a-crop}{12}{13}{0.96}{<++>}{<++>}{<++>}
\psd{week5a-crop}{14}{15}{0.96}{<++>}{<++>}{<++>}
\psd{week5a-crop}{16}{17}{0.96}{<++>}{<++>}{<++>}
\psd{week5a-crop}{18}{19}{0.96}{<++>}{<++>}{<++>}
\psd{week5a-crop}{20}{21}{0.96}{<++>}{<++>}{<++>}
\psd{week5a-crop}{22}{23}{0.96}{<++>}{<++>}{<++>}
\psd{week5a-crop}{24}{25}{0.96}{<++>}{<++>}{<++>}
\pss{week5a-crop}{26}{0.96}{<++>}{<++>}
\chapter{Introduction to classes}
\pss{week5b-crop}{1}{0.96}{<++>}{<++>}
\psd{week5b-crop}{2}{3}{0.96}{<++>}{<++>}{<++>}
\psd{week5b-crop}{4}{5}{0.96}{<++>}{<++>}{<++>}
\psd{week5b-crop}{6}{7}{0.96}{<++>}{<++>}{<++>}
\psd{week5b-crop}{8}{9}{0.96}{<++>}{<++>}{<++>}
\psd{week5b-crop}{10}{11}{0.96}{<++>}{<++>}{<++>}
\psd{week5b-crop}{12}{13}{0.96}{<++>}{<++>}{<++>}
\psd{week5b-crop}{14}{15}{0.96}{<++>}{<++>}{<++>}
\psd{week5b-crop}{16}{17}{0.96}{<++>}{<++>}{<++>}
\psd{week5b-crop}{18}{19}{0.96}{<++>}{<++>}{<++>}
\psd{week5b-crop}{20}{21}{0.96}{<++>}{<++>}{<++>}
\psd{week5b-crop}{22}{23}{0.96}{<++>}{<++>}{<++>}
\psd{week5b-crop}{24}{25}{0.96}{<++>}{<++>}{<++>}
\psd{week5b-crop}{26}{27}{0.96}{<++>}{<++>}{<++>}
\psd{week5b-crop}{28}{29}{0.96}{<++>}{<++>}{<++>}
\psd{week5b-crop}{30}{31}{0.96}{<++>}{<++>}{<++>}
\psd{week5b-crop}{32}{33}{0.96}{<++>}{<++>}{<++>}
\pss{week5b-crop}{34}{0.96}{<++>}{<++>}

\chapter{Operators, Function objects, more about templates}
\pss{week7-crop}{1}{0.96}{}{}
\psd{week7-crop}{2}{3}{0.96}{Special members}{Operators as functions}{\cht{operator}}
\psd{week7-crop}{4}{5}{0.96}{Assignment operators}{Symmetric operators}{\cht{assignment operators}\cht{symmetric operators}}
\psd{week7-crop}{6}{7}{0.96}{Extending classes with operators}{More comments about operators}{\cht{class operators}}
\psd{week7-crop}{8}{9}{0.96}{Conversion operators}{Array subscript operator: \texttt{operator[]}}{\texttt{operator[]}}
\psd{week7-crop}{10}{11}{0.96}{Pointer operators: \texttt{operator*} and \texttt{operator->}}{The function call operator: \texttt{operator()}}{\texttt{operator*}\texttt{operator-/}\cht{operator()}}
\psd{week7-crop}{12}{13}{0.96}{Template specialization}{Trait types}{\cht{template specialization}\cht{trait types}}
\psd{week7-crop}{14}{15}{0.96}{typename}{Traits types (continued)}{}
\psd{week7-crop}{16}{17}{0.96}{Old style traits}{New style traits}{}
\psd{week7-crop}{18}{19}{0.96}{Another application of traits}{An automatic solution forall integral types}{\cht{helper}\cht{question}}
\psd{week7-crop}{20}{21}{0.96}{Procedural programming}{Generic programming}{}
\pss{week7-crop}{22}{0.96}{Function objects}{\cht{function objects}}

\chapter{Algorithms and Data Structures in C++}
\pss{week8-crop}{1}{0.96}{}{}
\psd{week8-crop}{2}{3}{0.96}{Complexity analysis}{The O notation}{\tht{worst case analysis}\tht{best case analysis}\tht{average case analysis}\tht{amortized analysis}\tht{$\mathcal{O}$}}
\psd{week8-crop}{4}{5}{0.96}{Notations}{Time assuming 1 billion operations per second ( 1Gop)}{\tht{$\Omega$}\tht{$\Theta$}}
\psd{week8-crop}{6}{7}{0.96}{Time assuming 10 petaoperations per second (10 Pop/s)}{Which algorithm do you prefer?}{\tht{Algorithm}}
\psd{week8-crop}{8}{9}{0.96}{Complexity: example 1}{Complexity: example 2}{}
\psd{week8-crop}{10}{11}{0.96}{Complexity: example 3}{Complexity: example 4}{}
\psd{week8-crop}{12}{13}{0.96}{Complexity: adding to an array (simple way)}{Complexity: adding to an array (clever way)}{\tht{question}}
\psd{week8-crop}{14}{15}{0.96}{STL: Standard Template Library}{The standard C++ library}{\cht{standard library}}
\psd{week8-crop}{16}{17}{0.96}{The \texttt{string} and \texttt{wstring} classes}{The \texttt{pair} template}{\cht{\texttt{string}}\cht{\texttt{wstring}}\cht{\texttt{pair} template}}
\psd{week8-crop}{18}{19}{0.96}{Data structures in C++}{The array or vector data structure}{\cht{data structures}\cht{vector}\cht{array}}
\psd{week8-crop}{20}{21}{0.96}{Slow $\mathcal{O}(N)$ insertion and removal in an array}{Fast $\mathcal{O}(1)$ removal and insertion at the end of an array}{}
\psd{week8-crop}{22}{23}{0.96}{The deque data structure (double ended queue)}{The stack data structure}{\cht{data structure deque}\cht{data structure stack}}
\psd{week8-crop}{24}{25}{0.96}{The queue data structure}{The priority queue data structure}{\cht{data structure queue}\cht{data structure priority queue}}
\psd{week8-crop}{26}{27}{0.96}{The linked list data structure}{The tree data structures}{\cht{data structure linked list}\cht{data structure tree}}
\psd{week8-crop}{28}{29}{0.96}{A node in a binary tree}{A binary tree}{\cht{binary tree}}
\psd{week8-crop}{30}{31}{0.96}{Unbalanced trees}{Tree data structures in the C++ standard}{\cht{\texttt{set}}\cht{\texttt{multiset}}\cht{\texttt{map}}\cht{\texttt{multimap}}}
\psd{week8-crop}{32}{33}{0.96}{The container concept in the C++ standard}{Connecting Algorithms to Sequences}{\cht{data structure container}}
\psd{week8-crop}{34}{35}{0.96}{}{}{}
\psd{week8-crop}{36}{37}{0.96}{F.T.S.E}{Iterators to the Rescue}{\tht{Fundamental THeorem of Software Engineering}}
\psd{week8-crop}{38}{39}{0.96}{Describe Concepts for std::find}{Traversing an array and a linked list}{\cht{std::find}}
\psd{week8-crop}{40}{41}{0.96}{$N\times M$ Algorithm Implementations?}{Generic traversal}{\cht{generic traversal}}
\psd{week8-crop}{42}{43}{0.96}{Implementing iterators for the array}{Implementing iterators for the linked list}{\cht{implement iterators}\cht{iterators}}
\psd{week8-crop}{44}{45}{0.96}{Iterators}{Container requirements}{\cht{container}}
\psd{week8-crop}{46}{47}{0.96}{Containers and sequences}{Sequence constructors}{\cht{sequence}}
\psd{week8-crop}{48}{49}{0.96}{Direct element access in deque and vector}{Inserting and removing at the beginning and end}{\cht{deque}\cht{vector}}
\psd{week8-crop}{50}{51}{0.96}{Inserting and erasing anywhere in a sequence}{Vector specific operations}{\cht{vector}}
\psd{week8-crop}{52}{53}{0.96}{The \texttt{valarray} template}{Sequence adapters: \texttt{queue} and \texttt{stack}}{\cht{\texttt{valarray}}\cht{\texttt{queue}}\cht{\texttt{stack}}}
\psd{week8-crop}{54}{55}{0.96}{\texttt{list}-specific functions}{The \texttt{map} class}{\cht{\texttt{list}}\cht{\texttt{map}}}
\psd{week8-crop}{56}{57}{0.96}{Other tree-like containers}{Search operations in trees}{}
\psd{week8-crop}{58}{59}{0.96}{Search example in a tree}{Almost Containers}{}
\psd{week8-crop}{60}{61}{0.96}{The generic algorithms}{Example: \texttt{find}}{\cht{\texttt{find}}}
\psd{week8-crop}{62}{63}{0.96}{Example: \texttt{find_if}}{Member functions as predicates}{\cht{\texttt{find_if}}<++>}
\psd{week8-crop}{64}{65}{0.96}{\texttt{push_back} and \texttt{back_inserter}}{Penna Population}{\cht{\texttt{push_back}}\cht{\texttt{back_inserter}}}
\psd{week8-crop}{66}{67}{0.96}{The binary search}{Example: lower_bound}{\tht{binary search}}
\psd{week8-crop}{68}{69}{0.96}{Algorithms overview}{}{\cht{algorithms overview}}
\psd{week8-crop}{70}{71}{0.96}{Exercise}{stream iterators and Shakespeare}{}
\pss{week8-crop}{72}{0.96}{Summary}{}

\chapter{Inheritance, Exceptions, A C++ review: from modular to generic programming}
\pss{week10-crop}{1}{0.96}{<++>}{<++>}
\psd{week10-crop}{2}{3}{0.96}{<++>}{<++>}{<++>}
\psd{week10-crop}{4}{5}{0.96}{<++>}{<++>}{<++>}
\psd{week10-crop}{6}{7}{0.96}{<++>}{<++>}{<++>}
\psd{week10-crop}{8}{9}{0.96}{<++>}{<++>}{<++>}
\psd{week10-crop}{10}{11}{0.96}{<++>}{<++>}{<++>}
\psd{week10-crop}{12}{13}{0.96}{<++>}{<++>}{<++>}
\psd{week10-crop}{14}{15}{0.96}{<++>}{<++>}{<++>}
\psd{week10-crop}{16}{17}{0.96}{<++>}{<++>}{<++>}
\psd{week10-crop}{18}{19}{0.96}{<++>}{<++>}{<++>}
\psd{week10-crop}{20}{21}{0.96}{<++>}{<++>}{<++>}
\psd{week10-crop}{22}{23}{0.96}{<++>}{<++>}{<++>}
\psd{week10-crop}{24}{25}{0.96}{<++>}{<++>}{<++>}
\psd{week10-crop}{26}{27}{0.96}{<++>}{<++>}{<++>}
\psd{week10-crop}{28}{29}{0.96}{<++>}{<++>}{<++>}
\psd{week10-crop}{30}{31}{0.96}{<++>}{<++>}{<++>}
\psd{week10-crop}{32}{33}{0.96}{<++>}{<++>}{<++>}
\pss{week10-crop}{34}{0.96}{<++>}{<++>}

\chapter{Optimization and numerical libraries}
\pss{week11-crop}{1}{0.96}{}{}
\psd{week11-crop}{2}{3}{0.96}{Optimization}{Profiling}{\cht{profiling}}
\psd{week11-crop}{4}{5}{0.96}{Choice of data structures}{Example: The best data structure for Penna model}{\tht{Penna}}
\psd{week11-crop}{6}{7}{0.96}{Choice of algorithms}{The Strasse algorithm}{\cht{Strassen}\cht{matrix multiplication}}
\psd{week11-crop}{8}{9}{0.96}{Comparing matrix multiplication algorithms}{How to find the best algorithm?}{\cht{algorithm}\cht{best}}
\psd{week11-crop}{10}{11}{0.96}{How to optimize}{Optimization in assembly language}{\cht{assembly language}}
\psd{week11-crop}{12}{13}{0.96}{Example: counting leading zeroes in an integer}{Inline assembly statements}{\texttt{asm}}
\psd{week11-crop}{14}{15}{0.96}{Another example: long integers}{128 bit integers in \texttt{int128.C}}{}
\psd{week11-crop}{16}{17}{0.96}{A step in between: compiler intrinsics}{Helping the compiler optimize}{\cht{intrinsics}\cht{compiler intrinsics}}
\psd{week11-crop}{18}{19}{0.96}{Optimization options}{Homework}{\cht{compiler optimiziation options}}
\psd{week11-crop}{20}{21}{0.96}{Copy propagation (automatic)}{Constant folding (automatic)}{\cht{constant folding}\cht{copy propagation}}
\psd{week11-crop}{22}{23}{0.96}{Dead code removal}{Strength reduction}{\cht{strength reduction}\cht{dead code removal}}
\psd{week11-crop}{24}{25}{0.96}{Variable renaming}{Common subexpression elimination}{\cht{common subexpression eliminating (automatic)}\cht{variable renaming}}
\psd{week11-crop}{26}{27}{0.96}{Common subexpression elimination (manual)}{Loop invariant code motion (automatic)}{\cht{loop invariant code motion}\cht{common subexpression eliminating}}
\psd{week11-crop}{28}{29}{0.96}{Loop invariant code motion (manual)}{Induction variable simplification}{\cht{induction variable simplification}\cht{loop invariant code motion}}
\psd{week11-crop}{30}{31}{0.96}{Importance of variable simplification}{Loop unrolling}{\cht{loop unrolling}\cht{variable simplification}}
\psd{week11-crop}{32}{33}{0.96}{Partial loop unrolling}{Aiming for unit stride (sequential memory access)}{\cht{unit stride}\cht{sequential memory access}\cht{loop unrolling}}
\psd{week11-crop}{34}{35}{0.96}{In-cache matrix-matrix multiplication}{Out of cache matrix multiplications}{\cht{matrix multiplication}\cht{cache}\cht{unit stride}}
\psd{week11-crop}{36}{37}{0.96}{Libraries for linear algebra}{Calling Fortran from C++}{\cht{fortran}\cht{libraries}\cht{linear algebra}}
\psd{week11-crop}{38}{39}{0.96}{Calling Fortran: Example}{BLAS}{\cht{blas}\cht{fortran}}
\psd{week11-crop}{40}{41}{0.96}{ATLAS}{ATLAS benchmark}{\cht{ATLAS}}
\psd{week11-crop}{42}{43}{0.96}{ATLAS benchmark}{LAPACK}{\cht{LAPACK}}
\psd{week11-crop}{44}{45}{0.96}{LAPACK \& BLAS naming conventions}{Important LAPACK functions}{\cht{LAPACK}\cht{BLAS}}
\psd{week11-crop}{46}{47}{0.96}{FFTW}{Commercial libraries}{\cht{FFTW}\cht{libraries}}

\pss{week12-crop}{1}{0.96}{<++>}{<++>}
\psd{week12-crop}{2}{3}{0.96}{<++>}{<++>}{<++>}
\psd{week12-crop}{4}{5}{0.96}{<++>}{<++>}{<++>}
\psd{week12-crop}{6}{7}{0.96}{<++>}{<++>}{<++>}
\psd{week12-crop}{8}{9}{0.96}{<++>}{<++>}{<++>}
\psd{week12-crop}{10}{11}{0.96}{<++>}{<++>}{<++>}
\psd{week12-crop}{12}{13}{0.96}{<++>}{<++>}{<++>}
\psd{week12-crop}{14}{15}{0.96}{<++>}{<++>}{<++>}
\psd{week12-crop}{16}{17}{0.96}{<++>}{<++>}{<++>}
\psd{week12-crop}{18}{19}{0.96}{<++>}{<++>}{<++>}
\psd{week12-crop}{20}{21}{0.96}{<++>}{<++>}{<++>}
\psd{week12-crop}{22}{23}{0.96}{<++>}{<++>}{<++>}
\psd{week12-crop}{24}{25}{0.96}{<++>}{<++>}{<++>}
\psd{week12-crop}{26}{27}{0.96}{<++>}{<++>}{<++>}
\psd{week12-crop}{28}{29}{0.96}{<++>}{<++>}{<++>}
\psd{week12-crop}{30}{31}{0.96}{<++>}{<++>}{<++>}
\psd{week12-crop}{32}{33}{0.96}{<++>}{<++>}{<++>}
\psd{week12-crop}{34}{35}{0.96}{<++>}{<++>}{<++>}
\psd{week12-crop}{36}{37}{0.96}{<++>}{<++>}{<++>}

\chapter{An Introduction to Parallel Computing}
\pss{week13-crop}{1}{0.96}{<++>}{<++>}
\psd{week13-crop}{2}{3}{0.96}{<++>}{<++>}{<++>}
\psd{week13-crop}{4}{5}{0.96}{<++>}{<++>}{<++>}
\psd{week13-crop}{6}{7}{0.96}{<++>}{<++>}{<++>}
\psd{week13-crop}{8}{9}{0.96}{<++>}{<++>}{<++>}
\psd{week13-crop}{10}{11}{0.96}{<++>}{<++>}{<++>}
\psd{week13-crop}{12}{13}{0.96}{<++>}{<++>}{<++>}
\psd{week13-crop}{14}{15}{0.96}{<++>}{<++>}{<++>}
\psd{week13-crop}{16}{17}{0.96}{<++>}{<++>}{<++>}
\psd{week13-crop}{18}{19}{0.96}{<++>}{<++>}{<++>}
\psd{week13-crop}{20}{21}{0.96}{<++>}{<++>}{<++>}
\psd{week13-crop}{22}{23}{0.96}{<++>}{<++>}{<++>}
\psd{week13-crop}{24}{25}{0.96}{<++>}{<++>}{<++>}
\psd{week13-crop}{26}{27}{0.96}{<++>}{<++>}{<++>}
\psd{week13-crop}{28}{29}{0.96}{<++>}{<++>}{<++>}
\psd{week13-crop}{30}{31}{0.96}{<++>}{<++>}{<++>}
\psd{week13-crop}{32}{33}{0.96}{<++>}{<++>}{<++>}
\psd{week13-crop}{34}{35}{0.96}{<++>}{<++>}{<++>}
\psd{week13-crop}{36}{37}{0.96}{<++>}{<++>}{<++>}
\psd{week13-crop}{38}{39}{0.96}{<++>}{<++>}{<++>}
\psd{week13-crop}{40}{41}{0.96}{<++>}{<++>}{<++>}
\psd{week13-crop}{42}{43}{0.96}{<++>}{<++>}{<++>}
\psd{week13-crop}{44}{45}{0.96}{<++>}{<++>}{<++>}
\psd{week13-crop}{46}{47}{0.96}{<++>}{<++>}{<++>}
\psd{week13-crop}{48}{49}{0.96}{<++>}{<++>}{<++>}
\psd{week13-crop}{50}{51}{0.96}{<++>}{<++>}{<++>}
\psd{week13-crop}{52}{53}{0.96}{<++>}{<++>}{<++>}


\tableofcontents
\printindex[cpp]
\printindex[theory]
\printindex[git]

\end{document}

% LocalWords:  asimple
